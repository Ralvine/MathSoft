\section{算例分析}

由前述算法构建的测试算例源文件附于 src 目录,测试流程体现于 Shell 脚本。以下图像标题以 N , ox , oy , Deminsion , creal , cimag 的顺序标注参数数据,其中后两者仅限于 Julia 集。

\textbf{Step.1} 图像分析

\begin{figure}[htbp]
\centering
\begin{minipage}{0.33\linewidth}
\centering
\includegraphics[width = 5cm]{./images/julia2-1.bmp}
\caption{100,0,0,2,-0.4,0.6}
\label{fig2-1}
\end{minipage}\hfill
\begin{minipage}{0.33\linewidth}
\centering
\includegraphics[width = 5cm]{./images/julia2-2.bmp}
\caption{100,0,0,2,-0.6,-0.4}
\label{fig2-2}
\end{minipage}\hfill
\begin{minipage}{0.33\linewidth}
\centering
\includegraphics[width = 5cm]{./images/julia2-3.bmp}
\caption{100,0,0,2,-0.8,0.16}
\label{fig2-3}
\end{minipage}
\end{figure}

首先输入不同参数$c$,成功生成图像不同条件下的 Julia 集图像。其中,


\textbf{Step.2} 算法检验

\begin{figure}[htbp]
\centering
\begin{minipage}{0.33\linewidth}
\centering
\includegraphics[width = 5cm]{./images/julia3-1.bmp}
\caption{40,0,0,2,-0.4,0.6}
\label{fig3-1}
\end{minipage}\hfill
\begin{minipage}{0.33\linewidth}
\centering
\includegraphics[width = 5cm]{./images/julia3-2.bmp}
\caption{100,0,0,1,-0.6,-0.4}
\label{fig3-2}
\end{minipage}\hfill
\begin{minipage}{0.33\linewidth}
\centering
\includegraphics[width = 5cm]{./images/julia3-3.bmp}
\caption{100,0.6,0.5,2,-0.8,0.16}
\label{fig3-3}
\end{minipage}
\end{figure}

图\ref{fig2-1}与前文示例图\ref{fig1-3}的参数$c$相同,得到的结果也基本一致。

\textbf{Step.3} 特性比较

\begin{figure}[htb]
\centering
\begin{minipage}{0.45\linewidth}
\centering
\includegraphics[width = 5cm]{./images/julia4-1.bmp}
\caption{Julia,100,0,0,3,0.25,0}
\label{fig4-1}
\end{minipage}\hfill
\begin{minipage}{0.45\linewidth}
\centering
\includegraphics[width = 5cm]{./images/mandel4-3.bmp}
\caption{Mandelbrot,100,0.4,0,1}
\label{fig4-2}
\end{minipage}
\end{figure}
