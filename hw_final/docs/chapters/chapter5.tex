\section{结论}
自此,本文在引入 Mandelbrot 集作为参照的情形下介绍了 Julia 集的背景、原理与算法实现方式,成功将迭代公式付诸计算机图像绘制的实践,探索了 Julia 图像的诸多特性。

总体而言,本文得出了以下结论:Julia 集合图像具有中心对称的特性;Mandelbrot 集合本质是所有参数$c$情形的 Julia 集合的全体;二者使用相同且简洁的复二次多项式作为迭代意义上的基本定义,只是迭代的参数有所不同;二者都具有无限精细、自相似的特性;当$c$的取值在 Mandelbrot 集合内部时,Julia 集合是连通的;反之同理。

通过算例的分析与比较,相信读者对二者所代表的分形几何理论有了更为深入的感知。我们可以明晰 Julia 集与 Mandelbrot集在迭代方式上的共性,认识到此种自相似性在自然界中的普遍存在与美学价值,也或许会惊叹于自然界的纷繁被如此简洁公式所极尽刻画的奇妙。