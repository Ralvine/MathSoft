\section{算法}

\subsection{基本思路}

设计算法时,我们一般采用下述收敛判断方法以在有限的运算量中不断逼近 Julia 集。

我们通常假设$|c|<2$,那么经过$k$次迭代后,倘若$|z|>2$,这意味着
$|z^2|=|z|^2>2\cdot|z|$,显然,$|z|$自此随着迭代越来越大,即该点发散。反之,这样的点可以构成
的模抵消到原来的水平。因此,在迭代运算过程中,一旦某一步结果的模大于 2 了,可以断定它必将发散到无穷。\textsuperscript{\cite{website-julia-mandelbrot}}

但在实际计算中,我们自然无法找出所有的收敛点,但可以采取有限逼近的方法寻找所有模不大于 2 的点集;而由于图像的精度是有限的,最终可以转化为对平面网格上所有点进行收敛判定$\rightarrow$填充的一般算法,伪代码如下:

\begin{lstlisting}
R = escape radius  # R > 0 && R^2 - R >= sqrt(cx^2 + cy^2)
For Each Pixel(x,y) on the screen:
    zx = scaled x coordinate of pixel;
    zy = scaled y coordinate of pixel;
    iteration = 0;
    max_iteration = 1000;
    while (zx * zx + zy * zy < R^2  &&  iteration < max_iteration)
        xtemp = zx * zx - zy * zy;
        zy = 2 * zx * zy  + cy;
        zx = xtemp + cx;
        iteration = iteration + 1;
    if (iteration == max_iteration): let the pixel be black;
    else: return iteration;
Finish drawing pixels, output the bmp file.
\end{lstlisting}

\subsection{测试算法}

为了更为自由地呈现生成图片的细节以便于测试,算法中还设计了一些可调的参数:

\begin{itemize}
    \item 迭代次数\verb|N|:控制最大迭代次数,理论上数值越大,图像越精细。
    \item 图像中心坐标\verb|ox,oy|:控制生成图像在画布上的中心位置。
    \item 图像半径尺寸\verb|Dimension|:实现缩放效果。
    \item Julia 迭代参数$c$的实部和虚部:控制该参数,改变生成图像形态。
\end{itemize}