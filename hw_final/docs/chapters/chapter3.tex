\section{算法}

设计算法时,我们一般采用下述收敛判断方法以在有限的运算量中不断逼近 Julia 集。

我们通常假设$|c|<2$,那么经过$k$次迭代后,倘若$|z|>2$,这意味着
$|z^2|=|z|^2>2\dot|z|$,显然,$|z|$自此随着迭代越来越大,即该点发散。反之,这样的点可以构成
的模抵消到原来的水平。因此,在迭代运算过程中,一旦某一步结果的模大于 2 了,可以断定它必将发散到无穷。\textsuperscript{\cite{website-julia-mandelbrot}}

但在实际计算中,我们自然无法找出所有的收敛点,但可以采取有限逼近的方法寻找所有模不大于 2 的点集;而由于图像的精度是有限的,最终可以转化为对平面网格上所有点进行收敛判定$\rightarrow$填充的一般算法,伪代码如下:

\begin{verbatim}
R = escape radius  #choose R > 0 such that R**2 - R >= sqrt(cx**2 + cy**2)
for each pixel (x, y) on the screen, do:
    zx = scaled x coordinate of pixel # (scale to be between -R and R)
       # zx represents the real part of z.
    zy = scaled y coordinate of pixel # (scale to be between -R and R)
       # zy represents the imaginary part of z.
    iteration = 0
    max_iteration = 1000
    while (zx * zx + zy * zy < R^2  &&  iteration < max_iteration) 
    {
        xtemp = zx * zx - zy * zy
        zy = 2 * zx * zy  + cy 
        zx = xtemp + cx
        iteration = iteration + 1 
    }
    if (iteration == max_iteration)
        return black;
    else
        return iteration;
}
\end{verbatim}